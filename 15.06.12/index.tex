\documentclass[14pt]{article}
\renewcommand{\baselinestretch}{1.5}
%\documentclass[14pt,a4paper]{PhDthesis}
\usepackage[utf8]{inputenc}
\usepackage[russian]{babel}
\usepackage{amsmath}
\usepackage{amssymb}
\usepackage{fullpage}
\usepackage{indentfirst}
\usepackage{graphicx}
\usepackage{tabularx,multirow}
\usepackage{psfrag}
\usepackage{cite}

\usepackage{listings}
\usepackage{ucs}
\usepackage{listings}
\usepackage{amsmath}
\usepackage{tabularx}
\usepackage{hyperref}
\usepackage{tikz}
\usetikzlibrary{positioning,arrows}
\usepackage{graphicx}
\textwidth=16cm
\usepackage{amssymb}
\usepackage{fullpage}
\usepackage{indentfirst}
\usepackage{graphicx}
\usepackage{psfrag}
\usepackage{cite}
\usepackage{fullpage}
%\usepackage[dvips]{graphicx}
%\graphicspath{{images/}}
\lstset{inputencoding=utf8x, extendedchars=\true, captionpos=b, tabsize=3, keywordstyle=\color{blue},commentstyle=\color{green}, stringstyle=\color{red}, showstringspaces=false, basicstyle=\footnotesize,emph={label}, texcl}
%\lstset{texcl}
%\lstset{extendedchars=\true}
\lstset{
literate={а}{{\selectfont\char224}}1
{б}{{\selectfont\char225}}1
{в}{{\selectfont\char226}}1
{г}{{\selectfont\char227}}1
{д}{{\selectfont\char228}}1
{е}{{\selectfont\char229}}1
{ё}{{\"e}}1
{ж}{{\selectfont\char230}}1
{з}{{\selectfont\char231}}1
{и}{{\selectfont\char232}}1
{й}{{\selectfont\char233}}1
{к}{{\selectfont\char234}}1
{л}{{\selectfont\char235}}1
{м}{{\selectfont\char236}}1
{н}{{\selectfont\char237}}1
{о}{{\selectfont\char238}}1
{п}{{\selectfont\char239}}1
{р}{{\selectfont\char240}}1
{с}{{\selectfont\char241}}1
{т}{{\selectfont\char242}}1
{у}{{\selectfont\char243}}1
{ф}{{\selectfont\char244}}1
{х}{{\selectfont\char245}}1
{ц}{{\selectfont\char246}}1
{ч}{{\selectfont\char247}}1
{ш}{{\selectfont\char248}}1
{щ}{{\selectfont\char249}}1
{ъ}{{\selectfont\char250}}1
{ы}{{\selectfont\char251}}1
{ь}{{\selectfont\char252}}1
{э}{{\selectfont\char253}}1
{ю}{{\selectfont\char254}}1
{я}{{\selectfont\char255}}1
{А}{{\selectfont\char192}}1
{Б}{{\selectfont\char193}}1
{В}{{\selectfont\char194}}1
{Г}{{\selectfont\char195}}1
{Д}{{\selectfont\char196}}1
{Е}{{\selectfont\char197}}1
{Ё}{{\"E}}1
{Ж}{{\selectfont\char198}}1
{З}{{\selectfont\char199}}1
{И}{{\selectfont\char200}}1
{Й}{{\selectfont\char201}}1
{К}{{\selectfont\char202}}1
{Л}{{\selectfont\char203}}1
{М}{{\selectfont\char204}}1
{Н}{{\selectfont\char205}}1
{О}{{\selectfont\char206}}1
{П}{{\selectfont\char207}}1
{Р}{{\selectfont\char208}}1
{С}{{\selectfont\char209}}1
{Т}{{\selectfont\char210}}1
{У}{{\selectfont\char211}}1
{Ф}{{\selectfont\char212}}1
{Х}{{\selectfont\char213}}1
{Ц}{{\selectfont\char214}}1
{Ч}{{\selectfont\char215}}1
{Ш}{{\selectfont\char216}}1
{Щ}{{\selectfont\char217}}1
{Ъ}{{\selectfont\char218}}1
{Ы}{{\selectfont\char219}}1
{Ь}{{\selectfont\char220}}1
{Э}{{\selectfont\char221}}1
{Ю}{{\selectfont\char222}}1
{Я}{{\selectfont\char223}}1
}


 
\begin{document}
\thispagestyle{empty}

\section{Вычисления на основных энергетических состояниях электронов в квантовых точках}

Одним из перспективных применений полупроводниковых квантовых точек
является их использование в качестве элементов квантовой логики. Для выполнения
квантовых вычислений необходимо, чтобы задействованные квантовые состояния
обладали достаточно большим временем когерентности. Эксперименты [ссылки]
показывают, что при низких температурах в квантовых точках могут реализовываться
достаточно большие времена когерентности.

Квантовая точка --- это фрагмент проводника или полупроводника, содержащий в себе электрон или электроны, ограниченные со всех сторон каким-либо потенциальным барьером, в качестве барьера довольно часть применяют примеси каких-либо других веществ. Потенциал, создаваемый в квантовой точке,
ограничивает движение носителей заряда во всех трёх пространственных измерениях. Характерные размеры квантовой точки от 10 нм до 100 нм. Энергетический спектр такой системы дискретен.


Абсолютная величина момента собственного вращательного движения электрона составляет $\hbar S$, где $S = 1/2$ --- спин электрона. Момент вращательного движения является вектором. Проекция момента импульса может принимать только дискретные значения с интервалом $\hbar$. Поскольку $S=1/2$, то проекция момента собственного вращательного движения электрона может равняться лишь $-\hbar/2$ и $+\hbar/2$. 
 	
 

Так как электрон обладает зарядом, поэтому, подобно тому как вращение классического заряженного тела ведёт к появлению у него магнитного момента, собственное вращение (спин) индуцирует у электрона собственный магнитный момент $\mu$. Он направлен противоположно спину (так как заряд электрона отрицателен) и равен по абсолютной величине магнетону Бора $\mu_{B} = |e|\hbar/(2mc) \approx 0,927 \times 10^{-20}$ эрг/Гс. Проекция магнитного момента электрона на произвольно выбранную ось $z$ квантуется, то есть может принимать лишь два дискретных значения $\mu = -\mu_{B}$ ($S_{z} = 1/2$) и $\mu = \mu_{B}$ ($S_{z} = -1/2$).

Потенциальная энергия электрона в квантовой точке имеет локальный минимум, отделен-
ный энергетическим барьером от пространства, окружающего квантовую точку. Поэтому движение электрона в
квантовой точке ограничено во всех трех направлениях, и энергетический спектр является полностью дискрет-
ным, как в атоме.  Особенности электронного спектра полупроводников таковы, что все электроны занимают уровни энергии в низколежащей (так называемой валентной) энергетической зоне, тогда как уровни энергии в верхней энергетической зоне (зоне проводимости) вакантны.

Поскольку, изходя и принципа Паули,  на каждом уровне валентной зоны находится по два электрона
(один со спином вниз и один со спином вверх), то суммарная проекция $Sz$ спина всех электронов квантовой
точки равна нулю. Следовательно, равен нулю и магнитный момент квантовой точки. Если теперь в
квантовую точку добавить один лишний электрон.  В результате магнитный момент
квантовой точки станет отличен от нуля, а его направление (вниз или вверх) будет определяться направлением спина избыточного электрона. ($S_{z} = +1/2$ или $S_{z} = -1/2$ соответственно).

Энергия взаимодействия электронов в соседних квантовых точках зависит от взаимного направления спинов:
$E_{12} = J(S_{1}S_{2})$, где $S_{1}$ и $S_{2}$ --- векторы электронных спинов, величина $J$ зависит от геометрических характеристик и материалла квантовых точек, а также расстояния между ними.
Таким образом, $E_{12} = JS^{2}$, если векторы $S_{1}$ и $S_{2}$ параллельны, $E_{12} = -JS^{2}$ если они антипараллельны.

Пусть дана система из двух квантовых точек, понижая температуру данной системы до температуры близкой к 0 k (для простоты будем считать $T = 0$). Данная система займёт состояние с минимальной энергией, спинам двух соседних квантовых точек энергетически выгодно быть направленными в разные стороны. Если спин электрона в одной квантовой точке X направлен вниз, то спин электрона во второй квантовой точке  Y направлен вверх, и наоборот, то есть реализуются обе  строчки таблицы истинности вентиля «НЕ» (X = 0, Y = 1
и X = 1, Y = 0).












\newpage
\bibliographystyle{IzvAGU}
\bibliography{library}

\end{document}
